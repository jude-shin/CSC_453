\documentclass[11pt]{article}

% Use wide margins, but not quite so wide as fullpage.sty
\marginparwidth 0.5in 
\oddsidemargin 0.25in 
\evensidemargin 0.25in 
\marginparsep 0.25in
\topmargin 0.25in 
\textwidth 6in \textheight 8 in
% That's about enough definitions


\begin{document}
\hfill\vbox{\hbox{Shin, Jude}
		\hbox{CSC 453, Section 01}	
		\hbox{Lab 03}	
		\hbox{\today}}\par

\bigskip
\centerline{\Large\bf Lab 3: Problems}\par
\bigskip

\section*{Problem 1}
When the original file is removed, the linked "file" will still be present, 
however, when user 2 tries to read it, the command will look up that (old)
i-number in the i-nodes list. There won't be a file associated with it, so 
there will be an error. A link is not the same thing as copying a file; links
point to the same data. 


\section*{Problem 2}


\section*{Problem 3}


\section*{Problem 4}


\section*{Problem 5}


\section*{Problem 6}


\section*{Problem 7}
If you wanted to have a particular process be called by the scheduler twice as
frequent as the other processes due to priority or some other reason, then
no further modification would have to be done to the very simple scheduler.
Processes could be "seen" or processed more often because they appear more often
in the round robin list.


\section*{Problem 8}


\section*{Problem 9}


\section*{Problem 10}


\end{document}
