\documentclass[11pt]{article}

% Use wide margins, but not quite so wide as fullpage.sty
\marginparwidth 0.5in 
\oddsidemargin 0.25in 
\evensidemargin 0.25in 
\marginparsep 0.25in
\topmargin 0.25in 
\textwidth 6in \textheight 8 in
% That's about enough definitions


\begin{document}
\hfill\vbox{\hbox{Shin, Jude}
		\hbox{CSC 453, Section 01}	
		\hbox{Lab 02}	
		\hbox{\today}}\par

\bigskip
\centerline{\Large\bf Lab 2: {\sc Minix} Scavenger Hunt}\par
\bigskip
I am running {\sc Minix 3.1.8} through with Oracle's {\sc VirtualBox Manager}. The host machine is running vanilla {\sc Arch Linux}.

\setcounter{section}{0}
\section{Get a MINIX system}
\subsection{Approach}
The procedure is to choose a virtual machine, download a version of {\sc Minix} as a virtual hard disk file, and finally, configuring the virtual machine to use that virtual hard disk file. I chose {\sc VirtualBox}, as I already had it installed, and it was recommended. I downloaded the .gz file and used {\tt gunzip(1)} to decompress it. Finally, I went into the settings for the virtual machine I created, removed the auto-generated virtual hard disk, and replaced it with the .vmdk file that I decompressed. 

\subsection{Problems Encountered}
When I eventually found the location of the virtual hard disks in the VMs settings, I ran into issues with the controller compatibility for those virtual hard disks. I started getting errors when I would spin up the {\sc Minix} VM saying that the super block could not be read. 

{\tt\begin{tabbing}
Loading Boot Image 3.1.8.\\
kernel ds rs pm sched vfs memory log tty mfs vm pfs init (9626k)\\
\\
mount cannot read super block on /dev/c0d0p0s0: Error 0\\
mount: Can't mount /dev/c0d0p0s0 on /: Invalid argument\\
\end{tabbing}}

\subsection{Solutions}
I originally didn't change the default controller to an IDE controller (the default was SATA and at the time it seemed right to me). I did read earlier that we wanted the version of {\sc VirtualBox} with IDE-Controller, so I tried that out and it worked like a charm.

\subsection{Lessons Learned}
There were no lessons to be learned in this task. This was more of a hunt to find the right configuration.

\section{Log In}
\subsection{Approach}
The procedure is to just log into minix in any way. The only user that I assumed would be on a system would be root, as that is what most unix systems default to.

\subsection{Problems Encountered}
There were no problems encountered. Suprisingly, root had no password.

\subsection{Solutions}
There were no solutions, as there were no problems encountered.

\subsection{Lessons Learned}
There were no lessons to be learned in this task.

\section{Create a user account}
\subsection{Approach}
The procedure is to create a user account. I used {\tt adduser(8)} to create the user "jshin53", added it to the group "other", and set the home directory to be under "/home/jshin53". I then changed the password to a very secure password using {\tt passwd(1)}.

\subsection{Problems Encountered}
When I initally tried creating a user "jshin53", I assumed that that there would be a group named "users". There was not.

\subsection{Solutions}
The list of groups on the system are located in {\tt /etc/group}. I ran {\tt cat(1)} on the file and I just chose the group "other". The full command that I used was ``{\tt \$ adduser jshin53 other /home/jshin53}''.

\subsection{Lessons Learned}
I learned that a lot of the system wide files are in {\tt /etc}. I also learned that groups are one of the ways that unix systems manage the privilages users get.

\section{Create a {\sc Minix} disk image and use it to store data}
\subsection{Approach}
The procedure for this task is to first make an empty file to store the data as a pretend floppy disk. Then, register the virtual floppydisk through the {\sc VirtualBox}. Formatting the drive is optional in this case because we are using a virtual floppydisk. Finally, before mounting the floppy disk, formtat {\tt /dev/fd0} to a filesystem. Mount and unmount using the {\tt mount(1)} and {\tt umount} commands respectively.

\subsection{Problems Encountered}
I did not know how {\tt mkfs(1)} worked, even after reading the man pages.

\subsection{Solutions}
I read the footnotes. Suprisingly they are put there for a reason.

\subsection{Lessons Learned}
Formatting

\section{Accessing your data from outside {\sc Minix}}
\subsection{Approach}
The procedure follows like mounting any filesystem on a linux machine. {\tt mount(1)} the loopback device of type {\tt minix} on the virtual disk as {\tt /mnt/floppy}. To read or write to files in {\tt /mnt/floppy}, first {\tt umount(1)} it, and then access through the other device. The same procedure is used in minix as it is used in the host (linux) machine.

\subsection{Problems Encountered}
There were no problems encountered; the command was simple and laied out clearly in the lab manual.

\subsection{Solutions}
There were no solutions, as there were no problems encountered.

\subsection{Lessons Learned}
There were no lessons to be learned in this task. It was just a good refresher on how to mount drives on linux machines (like usb, external ssd, or floppydisks).

\section{Clean up and shut down}
\subsection{Approach}
The procedure is very trivial, as it is the same command that I use on my daily driver: {\tt shutdown(8)}. ``{\tt \$ shutdown now}'' will cleanly flush the cache and leave the machine in a neat state for when it is booted up again.

\subsection{Problems Encountered}
After using su to elevate my privilages, I used the command ``{\tt \$ shutdown now}. Afterwards, I was in this weird shell like state with a blinking cursor ``{\tt d0p0s0>}''. I could execute commands like {\tt ls} and {\tt help}, which made me think that I was not fully shutdown. I didn't know if it had to do with the virtual machine either.

\subsection{Solutions}
Professor Nico informed me thatk

\subsection{Lessons Learned}
There is a difference between

\end{document}
