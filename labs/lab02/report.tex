\documentclass[11pt]{article}

\title{CSC 453 Fall 2025: Labratory Exercise 2}
\author{Jude Shin}
\date{\today}

\begin{document}

\section*{Introduction}
I am running Minix 3.1.8 through an Oracle VirtualBox Manager with the host machine running vanilla Arch Linux.

\section*{Task 01: Get a MINIX system}

\subsection*{Approach}
The procedure is to choose a virtual machine, download a version of Minix as a virtual hard disk file, and finally, configuring the virtual machine to use that virtual hard disk file. I chose VirtualBox, as I already had it installed, and it was recommended. I downloaded the .gz file and unzipped it. Finally, I went into the settings for the virtual machine I created, removed the auto-generated virtual hard disk, and replaced it with the .vmdk file that I unzipped. 

\subsection*{Problems Encountered}
When I eventually found the location of the virtual hard disks in the VMs settings, I ran into issues with the controller compatibility for those virtual hard disks. I started getting errors when I would spin up the Minix VM saying that the super block could not be read. 

\begin{verbatim}
Loading Boot Image 3.1.8.
kernel ds rs pm sched vfs memory log tty mfs vm pfs init (9626k)

mount cannot read super block on /dev/c0d0p0s0: Error 0
mount: Can't mount /dev/c0d0p0s0 on /: Invalid argument
\end{verbatim}

\subsection*{Solutions}
I originally didn't change the default controller to an IDE controller (the default was SATA and at the time it seemed right to me). I did read earlier that we wanted the version of VirtualBox with IDE-Controller, so I tried that out and it worked like a charm.

\subsection*{Lessons Learned}
There were no lessons to be learned in this task. This was more of a hunt to find the right configuration.

\section*{Task 02: Log In}
\subsection*{Approach}
The procedure is to just log into minix in any way. The only user that I assumed would be on a system would be root, as that is what most unix systems default to.

\subsection*{Problems Encountered}
There were no problems encountered. Suprisingly, root had no password.

\subsection*{Solutions}
There were no solutions, as there were no problems encountered.

\subsection*{Lessons Learned}
There were no lessons to be learned in this task.

\section*{Task 03: Create a User Account}
\subsection*{Approach}
The procedure is to create a user account. I used 'adduser' to create the user "jshin53", added it to the group "other", and set the home directory to be under /home/jshin53. I then changed the password to a very secure password using the 'passwd' command.

\subseciton*{Problems Encountered}
When I initally tried creating a user "jshin53", I assumed that that there would be a group named "users". There was not.

\subsection*{Solutions}
The list of groups on the system are located in /etc/group. I ran 'cat' on the file and I just chose the group "other". The full command that I used was 'adduser jshin53 other /home/jshin53'

\subsection*{Lessons Learned}
I learned that a lot of the system wide files are in /etc. I also learned that groups are one of the ways that unix systems manage the privilages users get.


\section*{Task 04: Create a Minix Disk Image and Use it to Store Data}
\subsection*{Approach}
The procedure for this task is to first make an empty file to store the data as a pretend floppy disk. Then, register the virtual floppydisk through the VM of choice. Formatting the drive is optional in this case because we are using a virtual floppydisk. Finally, before mounting the floppy disk, formtat /dev/fd0 to a filesystem. Mount and unmount using the 'mount' and 'umount' commands respectively.

\subseciton*{Problems Encountered}
I did not know how mkfs worked, even after reading the man pages.

\subsection*{Solutions}
I read the footnotes. Suprisingly they are put there for a reason.

\subsection*{Lessons Learned}
Formatting

\section*{Task 05: Accessing Your Data from Outside Minix}
\subsection*{Approach}
The procedure follows like mounting any filesystem on a linux machine. Mount the loopback device of type "minix" on the virtual disk as /mnt/floppy. Read or write to files in /mnt/floppy, unmount it, and then access through the other device. The same procedure is used in minix as it is used in the host (linux) machine.

\subseciton*{Problems Encountered}
There were no problems encountered; the command was simple and laied out clearly in the lab manual.

\subsection*{Solutions}
There were no solutions, as there were no problems encountered.

\subsection*{Lessons Learned}
There were no lessons to be learned in this task. It was just a good refresher on how to mount drives on linux machines (like usb, external ssd, or floppydisks).

\section{Task 06: Clean Up and Shut Down}
\subsection*{Approach}
The procedure is very trivial, as it is the same command that I use on my daily driver: shutdown(8). 'shutdown now' will cleanly flush the cache and leave the machine in a neat state for when it is booted up again.

\subseciton*{Problems Encountered}
After using su to elevate my privilages, I used the command shutdown now. Afterwards, I was in this weird shell like state with a blinking cursor: d0p0s0>. I could execute commands like ls and stuff, which made me think that I was not fully shutdown. I didn't know if it had to do with the virtual machine either.

\subsection*{Solutions}
Professor Nico informed me thatk

\subsection*{Lessons Learned}
There is a difference between

\end{document}
